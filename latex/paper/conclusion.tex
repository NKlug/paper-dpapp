 Wir haben uns damit befasst, einen gegebenen Kantenzug $P$ so zu approximieren, dass die Länge der Approximation nur um einen bestimmten Faktor $t$ von der von $P$ abweicht. 
Im Rahmen dieser Zielsetzung haben wir zwei Probleme betrachtet: Die Minimierung der Knotenzahl der Approximation bei gegebenem $t$ (MVPS) und die Minimierung von $t$ bei gegebener Knotenzahl (MDPS).
Dazu haben wir zunächst einen exakten Algorithmus für das MVPS-Problem betrachtet, der eine Laufzeit von $O(n^2)$ aufweist, und anschließend daraus einen exakten Algorithmus für das MDPS-Problem konstruiert, dessen Ausführung $O(n^2 \log n)$ Zeit kostet.

Die danach vorgestellten approximativen Algorithmen weisen dagegen eine deutlich bessere Laufzeit auf.
Das MVPS-Problem können wir dabei bis auf ein $\epsilon$ genau in $O(n \log n + \frac{t}{\epsilon}n)$ lösen.
Auch hier haben wir dieses Ergebnis verwendet, um daraus einen - diesmal approximativen - Algorithmus für das MDPS-Problem zu konstruieren. Dieser weist eine asymptotische Laufzeit von $O(\frac{t^*}{\epsilon}n \cdot \log n)$ auf, wobei $t^*$ die exakte Lösung ist.

Die Autoren des Artikels \cite{gudmundsson}, auf dem diese Arbeit beruht, zeigten 2006 in einem Experiment, dass die Laufzeit der approximativen Algorithmen nicht nur asymptotisch, sondern auch praktisch die der exakten Algorithmen deutlich unterbietet.
So gelang die exakte Lösung des MVPS-Problems für einen Kantenzug mit $20.000$ Punkten und einer Abweichung $t = 1.1$ in etwa 97 Sekunden, während der approximative Algorithmus für das selbe $t$ und $\epsilon = 0.05$ nur 7.2 Sekunden brauchte.