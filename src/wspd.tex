\begin{definition}[wohl-separiert]
    	\label{def:wellsep}
    	Sei $s > 0$ und $A$ und $B$ zwei endliche Mengen von Punkten im $\R^d$. $A$ und $B$ heißen wohl-separiert in Bezug zu $s$ (engl. well-separated), falls es zwei disjunkte Bälle $C_A$ und $C_B$ gibt, die denselben Radius $R$ haben, wobei $A \subseteq C_A$ und $B \subseteq C_B$ und die euklidische Distanz zwischen $C_A$ und $C_B$ mindestens $s\cdot R$ beträgt.
    \end{definition}
    
    Das folgende Lemma hält zwei wichtige Eigenschaften von zwei wohl-separierten Mengen $A$ und $B$ fest.
    \begin{lemma}
    	\label{lem:wellsep}
		Seien $a, a' \in A$ und $b, b' \in B$. Dann gilt:
		\begin{enumerate}
			\item $\displaystyle |aa'| \leq \frac{2}{s}\cdot|a'b'|$
			\item $\displaystyle |a'b'| \leq (1+\frac{4}{s})\cdot|ab|$
		\end{enumerate}
    \end{lemma}
    \begin{proof}
    	Zu 1. Ist $r$ der Radius von $C_A$ und $C_B$, so gilt $|aa'| \leq 2 \cdot r$. Da $A$ und $B$ wohl-separiert sind, gilt $|a'b'| \geq s \cdot r$, was äquivalent ist zu $r \leq \frac{|a'b'|}{s}$. Durch Einsetzen folgt dann die Behauptung.
    	Zu 2. Da $A$ und $B$ wohl-separiert in Bezug zu $s$ sind, und $C_A$ und $C_B$ beide denselben Radius $r$ haben, gilt $|a'b'| \leq s \cdot r + 4 \cdot r$. Ausklammern rechts ergibt $(1 + \frac{4}{s}) \cdot s \cdot r$. Da ja auch $s \cdot r \leq |ab|$, folgt durch Einsetzen die Behauptung.
    \end{proof}
    
    \begin{definition}[well-separated pair decomposition]
    	\label{def:wspd}
    	Sei $S \subseteq \R^d$ und $s > 0$. Eine Folge $(A_i, B_i)_{1 \leq i \leq m}$ von nicht-leeren Teilmengen von S ist genau dann eine \emph{Zerlegung in wohl-separierte Paare (engl. well-separated pair decomposition; WSPD)}, wenn gilt:
    	\begin{enumerate}[label={(\arabic*)}, itemsep=0mm]
    		\item $A_i \cap B_i = \emptyset$
    		\item Für alle $p, q \in S$ gibt es genau einen Index $1 \leq i \leq m$, sodass entweder $p \in A_i$ und $q \in B_i$ oder $q \in A_i$ und $p \in B_i$.
    		\item $A_i$ und $B_i$ sind wohl-separiert in Bezug zu $s$
    	\end{enumerate}
    \end{definition}
	
	\noindent $m$ nennen wir dabei die \emph{Größe} der WSPD.

	Die WSPD bildet eine wichtige Grundlage für die beiden Algorithmen, die wir im Folgenden betrachten werden. 
	\textellipsis haben gezeigt, dass man eine WSPD der Größe $m = O(n)$ in $O(n\log n)$ Zeit berechnen kann. 
	Dabei wird zunächst ein sogenannter \emph{fairer Split Tree} berechnet, aus dem dann in $O(s^dn)$ Zeit eine WSPD erstellt werden kann. 
	Wir werden sehen, dass es für unseren Anwendungsfall genügt, eine WSPD für Mengen von Punkten aus $\R$ zu erstellen. 
	Für diesen 1-dimensionalen Fall kann der faire Split Tree mit Hilfe eines einfachen Algorithmus berechnet werden.
	
	\begin{figure}[b]
	\centering
	\begin{minipage}{.8\linewidth}
		\scriptsize
		\begin{algorithmic}[H]
			\STATE \texttt{compute\_split\_tree(i, j)}  \{
			\begin{ALC@g}
				\IF{$i = j$}
					\STATE erstelle neuen Knoten $u$;
					\STATE speichere das Intervall $[i,i]$ zu $u$;
					\RETURN $u$
				\ELSE
					\STATE $z \coloneqq (S[i] + S[j]) / 2$;
					\STATE $k \coloneqq \text{Index eines Elementes von } S \text{, sodass } S[k] \leq z < S[k+1]$;
					\STATE $v \coloneqq \texttt{compute\_split\_tree(i, k)}$;
					\STATE $w \coloneqq \texttt{compute\_split\_tree(k+1, j)}$;
					\STATE erstelle neuen Knoten $u$;
					\STATE speichere das Intervall $[i, j]$ zu $u$;
					\STATE mache $v$ zum linken Kind von $u$;
					\STATE mache $w$ zum rechten Kind von $u$;
					\RETURN $u$
				\ENDIF
			\end{ALC@g}
			\STATE \}
		\end{algorithmic}
	\end{minipage}
	\caption{Algorithmus zum Erstellen eines fairen Split-Trees zu einer gegebenen Menge $S$ (nach \cite{gudmundsson})}
	\label{fig:splittree}
\end{figure}


	
	Sei $S'$ eine endliche Teilmenge von $\R$ und $|S'| = n$. Wir können davon ausgehen, dass uns diese Menge sortiert in einem Array $S[1..n]$ vorliegt und werden später sehen, dass das bei unserem Algorithmus auch tatsächlich der Fall ist. Abbildung \ref{fig:splittree} stellt einen Algorithmus dar, der diesen Split Tree $T$ erstellt. Bei $T$ handelt es sich um einen Binärbaum, an dessen Blättern die Werte von $S$ in von links nach rechts aufsteigend sortierter Reihenfolge gespeichert sind. Für jeden inneren Knoten wird zusätzlich das Intervall in dem die Blätter des von ihm induzierten Teilbaumes liegen gepsiechert.
	Da $T$ $n$ Blätter hat, erstellen wir $O(n)$ Knoten. Dabei müssen wir aber in ZEILE 88 jedesmal eine Binärsuche durchführen, die $O(\log n)$ Zeit kostet. Somit ergibt sich für das Erstellen von $T$ eine Gesamtlaufzeit von $O(n\log n)$.
	Betrachten wir jetzt zwei innere Knoten $p$ und $q$ von T. Seien $[i, j]$ und $[k, l]$ die Intervalle, die wir mit $p$ und $q$ gespeichert haben und 
	\[R \coloneqq \max(i - j, k - l)\]
	Nach Definition \ref{def:wellsep} sind die beiden Intervalle genau dann wohl-separiert, wenn 
	\[k - j \geq R \cdot s \text{ oder } i - l \geq R \cdot s \]
	
	\begin{figure}
	\centering
	\begin{minipage}{.8\linewidth}
		\scriptsize
		\begin{algorithmic}[H]
			\STATE \texttt{compute\_wspd(T, s)}  \{
			\begin{ALC@g}
				\STATE \textbf{for each} innerer Knoten $u$ in $T$ \textbf{do}
				\begin{ALC@g}
					\STATE $v \coloneqq \text{ linkes Kind von } u$
					\STATE $w \coloneqq \text{ rechtes Kind von } u$
					\STATE \texttt{find\_pairs(v, w)}
				\end{ALC@g}
			\end{ALC@g}
			\STATE \}
			\STATE \
			\STATE \texttt{find\_pairs(v, w)} \{
			\begin{ALC@g}
				\IF{$S_u$ und $S_v$ sind in Bezug zu $s$ nicht wohl-separiert}
					\STATE Seien $[i, j]$ und $[k, l]$ die Intervalle die mit $u$ bzw. $v$ gespeichert sind;
						\IF{$S[j] - S[i] \leq S[l] - S[k]$}
							\STATE $w_1 \coloneqq \text{ linkes Kind von } w $;
							\STATE $w_2 \coloneqq \text{ rechtes Kind von } w $;
							\STATE \texttt{find\_pairs(v, w\textsubscript{1})};
							\STATE \texttt{find\_pairs(v, w\textsubscript{2})};
						\ELSE
							\STATE $v_1 \coloneqq \text{ linkes Kind von } v $;
							\STATE $v_2 \coloneqq \text{ rechtes Kind von } v $;
							\STATE \texttt{find\_pairs(v\textsubscript{1}, w)};
							\STATE \texttt{find\_pairs(v\textsubscript{2}, w)};
						\ENDIF
				\ELSE
					\STATE Speichere in $u$ und $v$, dass deren Blätter die Teilmengen $A$ und $B$ einer WSPD bilden;
				\ENDIF
			\end{ALC@g}
			
			\STATE \}
		\end{algorithmic}
	\end{minipage}
	\caption{Algorithmus zum Erstellen einer WSPD aus einem gegebenen Split-Tree $T$ und einer Trennungsrate $s$ (nach \cite{gudmundsson})}
	\label{fig:wspd}
\end{figure}
	
	Der in Abbildung \ref{fig:wspd} dargestellte Algorithmus berechnet dann aus dem fairen Split-Tree eine WSPD.
	Betrachten wir doch \texttt{compute\_wspd(T, s)} etwas genauer. Dabei werden für jeden Knoten $k$ dessen linke und rechte Kindknoten $u$ und $v$ betrachtet, und darauf \texttt{find\_pairs(u, v)} aufgerufen. Da die Elemente von $S$ in den Blättern gespeichert sind ist klar, dass die der linke Kindknoten $u$ und der rechte $v$ disjunkte Teilmengen von $S$ repräsentieren. Somit ist Forderung (1) einer WSPD erfüllt. \texttt{find\_pairs(u, v)} überprüft, ob die mit $u$ und $v$ gespeicherten Intervalle $S_u$ und $S_v$ wohl-separiert sind; ist dies der Fall, speichern wir mit $u$, dass seine Blätter das Element $A_i$ einer WSPD bilden, und mit $v$, dass seine Kinder das Element $B_i$ bilden. Sind die Intervalle nicht wohl-separiert, steigen wir solange in Richtung des größeren Intervalls im Baum herab, bis wir auf zwei wohl-separierte Intervalle treffen. Wir sehen also, dass die erste und die dritte Forderung der Definition der WSPD durch den Algorithmus erfüllt werden. Man kann auch zeigen, dass er die zweite erfüllt, was wir an dieser Stelle allerdings überspringen werden. Einen vollständigen Beweis kann man beispielsweise auf SEITE 88 in CITE X nachlesen. X und Y, die Autoren dieses Artikels, haben auch bewiesen, dass die Erstellung der zu $T$ gehörenden WSPD mit \texttt{compute\_wspd(T, s)} $O(sn)$ Zeit kostet. Halten wir also fest:
	\begin{theorem}
		\label{theo:wspdtime}
		Sei $S \subset \R$ endlich und $n = |S|$. Dann kann in $O(n \log n + sn)$ Zeit ein Split Tree $T$ und eine dazugehörige WSPD ${A_i, B_i}_{1 \leq i \leq m}$ der Größe $m = O(sn)$ berechnet werden.
	\end{theorem}