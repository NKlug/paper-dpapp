\begin{figure}
	\centering
	\begin{minipage}{.8\linewidth}
		\begin{algorithmic}[H]
			\STATE \texttt{compute\_split\_tree(i, j)}  \{
			\begin{ALC@g}
				\IF{$i = j$}
					\STATE erstelle neuen Knoten $u$;
					\STATE speichere das Intervall $[i,i]$ zu $u$;
					\RETURN $u$
				\ELSE
					\STATE $z \coloneqq (S[i] + S[j]) / 2$;
					\STATE $k \coloneqq \text{Index eines Elementes von } S \text{, sodass } S[k] \leq z < S[k+1]$;
					\STATE $v \coloneqq \texttt{compute\_split\_tree(i, k)}$;
					\STATE $w \coloneqq \texttt{compute\_split\_tree(k+1, j)}$;
					\STATE erstelle neuen Knoten $u$;
					\STATE speichere das Intervall $[i, j]$ zu $u$;
					\STATE mache $v$ zum linken Kind von $u$;
					\STATE mache $w$ zum rechten Kind von $u$;
					\RETURN $u$
				\ENDIF
			\end{ALC@g}
			\STATE \}
		\end{algorithmic}
	\end{minipage}
	\caption{Algorithmus zum Erstellen eines fairen Split-Trees zu einer gegebenen Menge $S$}
	\label{fig:splittree}
\end{figure}

